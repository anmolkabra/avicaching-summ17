\begin{appendices}
    \section{Implementation} \label{sec:Implementation}
    The code can be found here[]. \\
    Both the Identification and the Pricing Problem were programmed in Python 2.7 using NumPy 1.12.1, SciPy 0.19.1 and PyTorch 0.1.12 modules [web cites] \cite{SCPOptimizeDocs}\cite{NPDocs}. [Results from Python plotted in Matplotlib 2.0.2] With some code optimizations, the input dataset $\matr{F}$ was built using NumPy's \texttt{ndarray} and PyTorch's \texttt{tensor} functions. Since PyTorch offers NumPy-like code base but with dedicated neural network functions and submodules, PyTorch's \texttt{relu} and \texttt{softmax} functions were used along with other matrix operations.\\
    
    \subsection{Specific Implementation Details for the Pricing Problem}
    Among all the code optimizations in both models, some in that for the Pricing Problem are worth discussing, as they drastically differ from Algorithm \ref{alg:Solving the Pricing Problem} or are intricate. Most optimizations relevant to the Identification Problem are trivial and relate directly to those for the Pricing Problem. Therefore, only those in the Pricing Problem model are discussed.
    
    \subsubsection{Building the Dataset $\matr{F}$}
    Notice that we build the dataset $\matr{F}$ and batch-multiply it with $\matr{w_1}$ on each iteration/epoch (lines 2-3 of Algorithm \ref{alg:Solving the Pricing Problem}). Doing these steps are repetitive as most elements of $\matr{F}$, distances $\matr{D}$ and environmental feature vector $\vect{f}$, do not change unlike rewards $\vect{r}$. Moreover since $\matr{w_1}$ is fixed, Algorithm \ref{alg:Solving the Pricing Problem} would repetitively multiply the $\vect{f}$ and $\matr{D}$ components of $\matr{F}$ with $\matr{w_1}$. To avoid these unnecessary computations, we preprocessed most of $\matr{F}$ by batch-multiplying with $\matr{w_1}$ and only multiplied $\vect{r}$ with the corresponding elements of $\matr{w_1}$. Figure \ref{fig:Splitting and Batch Multiplying F and w1} describes the process graphically.\\
    \begin{figure}[!htbp]
        \centering
        \includegraphics[width=\linewidth]{split_and_batch_multiply}
        \caption{Splitting and Batch Multiplying $\matr{F}$ and $\matr{w_1}$}
        \label{fig:Splitting and Batch Multiplying F and w1}
    \end{figure}    
    Although this preprocessing might seem applicable for the model in Identification Problem too, it does not apply fully. Since the weights $\matr{w_1}$ are updated on each iteration/epoch, we cannot multiply them with parts of $\matr{F}$ beforehand (Algorithm \ref{alg:Algorithm for the Identification Problem}). However, we can combine $\matr{D}$ and $\vect{f}$ in the preprocessing stage and simply append $\vect{r}[t]$ on each iteration, saving computation time.
    
    \subsubsection{Modeling the Linear Programming Problem in the Standard Format}
    The \texttt{scipy.optimize} module's \texttt{linprog} function requires that the arguments are in standard LP format. As discussed in \cref{sec:Calculating Rewards}, Equation \ref{eq:lp_code_constrain_rewards} resembles the standard format more closely than \ref{eq:lp_math_constrain_rewards}, but it may not be clear how so.\\
    
    Considering $\vect{u}$ and $\vect{r'}$ as variables $\vect{x}$, Equation \ref{eq:lp_code_constrain_rewards} translates into Equation \ref{eq:lp_matrix_rewards} ($J$ is the number of locations).
    \begin{equation} \label{eq:lp_matrix_rewards}
    \begin{aligned}
    & \text{minimize}
    & & \begin{bmatrix}
    \vect{0_J}\\
    \vect{1_J}\\
    \end{bmatrix}^T
    \begin{bmatrix}
    \vect{r'}\\
    \vect{u}
    \end{bmatrix}\\ \\
    & \text{subject to}
    & & \begin{bmatrix}
    I_J & -I_J\\
    -I_J & -I_J\\
    \vect{1}^T_J & \vect{0}^T_J\\
    \end{bmatrix}
    \begin{bmatrix}
    \vect{r'}\\
    \vect{u}\\
    \end{bmatrix} \leq
    \begin{bmatrix}
    \vect{r}\\
    -\vect{r}\\
    \mathcal{R}\\
    \end{bmatrix}\\
    &&& r'_i, u_i \geq 0
    \end{aligned}
    \end{equation}
    
    \section{Strange GPU Speedup in LP Computation} \label{sec:Strange GPU Speedup in LP Computation}
    Even though we intentionally transferred the rewards vector to and constrained it using \texttt{scipy.optimize} module's \texttt{linprog} function on the CPU, we obtained an unexpected GPU speedup in the runtimes (see \cref{sec:PriProbRes - GPU} and Figure \ref{fig:Finding Rewards - Time taken by the LP}). Confounded by this weird behavior, we wanted to pinpoint the reasons because SciPy's function could not have differentiated between the configurations and delivered different results. However, since this was not our research's prime motive, we did not take a strong quantitative approach in determining the cause(s). There could have been many reasons for this bizarre behavior, including but not limited to:
    \begin{enumerate}
        \item SciPy's Optimize Module differentiating between configurations. This can be ruled out because the module could not have known the configuration during which it was called. This is because the configuration settings were applicable only on user-programmed operations, and needed to be explicitly stated - as mandated by PyTorch \cite{PTDocs}. SciPy's Optimize Module identifying the configurations is just supernatural.
        \item CPU ``set'' using exploiting more main memory than GPU ``set''. We suspected that since CPU ``set'' configuration's operations were executed solely on the CPU, the residing datasets could have used more main memory than when GPU ``set'' was running. This could have hampered the performance of LP with CPU ``set''. Unlike the 1\textsuperscript{st} possibility, this would have meant that CPU ``set'' was slowing down the LP, and not that GPU ``set'' was speeding up the LP.
    \end{enumerate}

    The 2\textsuperscript{nd} hypothesis/possibility seemed more promising than the others, and we built some programs to test its various aspects, which are elaborated in next sections.
    
    \subsection{Main Memory Usage by Both Configurations}
    As the initial approach, we logged details of main memory usage when the models were running. Contrary to our expectations (again), the GPU ``set'' was utilizing more main memory than CPU ``set'' was. Precisely, when running the Pricing Problem's model with $T = 173, J = 116$ without GUI and logging at 0.2 second intervals until the model completed, the GPU ``set'' was using 9.0 - 9.1 \% of the main memory, whereas CPU ``set'' was using only 0.9 \% memory. During the tests, only 1 CPU core was utilized for GPU ``set'', whereas 6-8 cores were being used for CPU ``set''. Clearly, main memory usage wouldn't have been the reason for the strange GPU speedup in LP runtime as LP's GPU ``set'' runtime would have been lower if the relation with main memory usage was true. However, there could be a relation between CPU usage and LP runtime.
    
    \subsubsection{asdf}
\end{appendices}